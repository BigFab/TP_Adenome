
% Default to the notebook output style

    


% Inherit from the specified cell style.




    
\documentclass[11pt]{article}
    
    \usepackage[T1]{fontenc}
    % Nicer default font (+ math font) than Computer Modern for most use cases
    \usepackage{mathpazo}

    % Basic figure setup, for now with no caption control since it's done
    % automatically by Pandoc (which extracts ![](path) syntax from Markdown).
    \usepackage{graphicx}
    \usepackage{subfig}
    
    % We will generate all images so they have a width \maxwidth. This means
    % that they will get their normal width if they fit onto the page, but
    % are scaled down if they would overflow the margins.
    \makeatletter
    \def\maxwidth{\ifdim\Gin@nat@width>\linewidth\linewidth
    \else\Gin@nat@width\fi}
    \makeatother
    \let\Oldincludegraphics\includegraphics
    % Set max figure width to be 80% of text width, for now hardcoded.
    \renewcommand{\includegraphics}[1]{\Oldincludegraphics[width=.8\maxwidth]{#1}}
    % Ensure that by default, figures have no caption (until we provide a
    % proper Figure object with a Caption API and a way to capture that
    % in the conversion process - todo).
    \usepackage{caption}
    \DeclareCaptionLabelFormat{nolabel}{}
    \captionsetup{labelformat=nolabel}

    \usepackage{adjustbox} % Used to constrain images to a maximum size 
    \usepackage{xcolor} % Allow colors to be defined
    \usepackage{enumerate} % Needed for markdown enumerations to work
    \usepackage{geometry} % Used to adjust the document margins
    \usepackage{amsmath} % Equations
    \usepackage{amssymb} % Equations
    \usepackage{textcomp} % defines textquotesingle
    % Hack from http://tex.stackexchange.com/a/47451/13684:
    \AtBeginDocument{%
        \def\PYZsq{\textquotesingle}% Upright quotes in Pygmentized code
    }
    \usepackage{upquote} % Upright quotes for verbatim code
    \usepackage{eurosym} % defines \euro
    \usepackage[mathletters]{ucs} % Extended unicode (utf-8) support
    \usepackage[utf8x]{inputenc} % Allow utf-8 characters in the tex document
    \usepackage{fancyvrb} % verbatim replacement that allows latex
    \usepackage{grffile} % extends the file name processing of package graphics 
                         % to support a larger range 
    % The hyperref package gives us a pdf with properly built
    % internal navigation ('pdf bookmarks' for the table of contents,
    % internal cross-reference links, web links for URLs, etc.)
    \usepackage{hyperref}
    \usepackage{longtable} % longtable support required by pandoc >1.10
    \usepackage{booktabs}  % table support for pandoc > 1.12.2
    \usepackage[inline]{enumitem} % IRkernel/repr support (it uses the enumerate* environment)
    \usepackage[normalem]{ulem} % ulem is needed to support strikethroughs (\sout)
                                % normalem makes italics be italics, not underlines
    

    
    
    % Colors for the hyperref package
    \definecolor{urlcolor}{rgb}{0,.145,.698}
    \definecolor{linkcolor}{rgb}{.71,0.21,0.01}
    \definecolor{citecolor}{rgb}{.12,.54,.11}

    % ANSI colors
    \definecolor{ansi-black}{HTML}{3E424D}
    \definecolor{ansi-black-intense}{HTML}{282C36}
    \definecolor{ansi-red}{HTML}{E75C58}
    \definecolor{ansi-red-intense}{HTML}{B22B31}
    \definecolor{ansi-green}{HTML}{00A250}
    \definecolor{ansi-green-intense}{HTML}{007427}
    \definecolor{ansi-yellow}{HTML}{DDB62B}
    \definecolor{ansi-yellow-intense}{HTML}{B27D12}
    \definecolor{ansi-blue}{HTML}{208FFB}
    \definecolor{ansi-blue-intense}{HTML}{0065CA}
    \definecolor{ansi-magenta}{HTML}{D160C4}
    \definecolor{ansi-magenta-intense}{HTML}{A03196}
    \definecolor{ansi-cyan}{HTML}{60C6C8}
    \definecolor{ansi-cyan-intense}{HTML}{258F8F}
    \definecolor{ansi-white}{HTML}{C5C1B4}
    \definecolor{ansi-white-intense}{HTML}{A1A6B2}

    % commands and environments needed by pandoc snippets
    % extracted from the output of `pandoc -s`
    \providecommand{\tightlist}{%
      \setlength{\itemsep}{0pt}\setlength{\parskip}{0pt}}
    \DefineVerbatimEnvironment{Highlighting}{Verbatim}{commandchars=\\\{\}}
    % Add ',fontsize=\small' for more characters per line
    \newenvironment{Shaded}{}{}
    \newcommand{\KeywordTok}[1]{\textcolor[rgb]{0.00,0.44,0.13}{\textbf{{#1}}}}
    \newcommand{\DataTypeTok}[1]{\textcolor[rgb]{0.56,0.13,0.00}{{#1}}}
    \newcommand{\DecValTok}[1]{\textcolor[rgb]{0.25,0.63,0.44}{{#1}}}
    \newcommand{\BaseNTok}[1]{\textcolor[rgb]{0.25,0.63,0.44}{{#1}}}
    \newcommand{\FloatTok}[1]{\textcolor[rgb]{0.25,0.63,0.44}{{#1}}}
    \newcommand{\CharTok}[1]{\textcolor[rgb]{0.25,0.44,0.63}{{#1}}}
    \newcommand{\StringTok}[1]{\textcolor[rgb]{0.25,0.44,0.63}{{#1}}}
    \newcommand{\CommentTok}[1]{\textcolor[rgb]{0.38,0.63,0.69}{\textit{{#1}}}}
    \newcommand{\OtherTok}[1]{\textcolor[rgb]{0.00,0.44,0.13}{{#1}}}
    \newcommand{\AlertTok}[1]{\textcolor[rgb]{1.00,0.00,0.00}{\textbf{{#1}}}}
    \newcommand{\FunctionTok}[1]{\textcolor[rgb]{0.02,0.16,0.49}{{#1}}}
    \newcommand{\RegionMarkerTok}[1]{{#1}}
    \newcommand{\ErrorTok}[1]{\textcolor[rgb]{1.00,0.00,0.00}{\textbf{{#1}}}}
    \newcommand{\NormalTok}[1]{{#1}}
    
    % Additional commands for more recent versions of Pandoc
    \newcommand{\ConstantTok}[1]{\textcolor[rgb]{0.53,0.00,0.00}{{#1}}}
    \newcommand{\SpecialCharTok}[1]{\textcolor[rgb]{0.25,0.44,0.63}{{#1}}}
    \newcommand{\VerbatimStringTok}[1]{\textcolor[rgb]{0.25,0.44,0.63}{{#1}}}
    \newcommand{\SpecialStringTok}[1]{\textcolor[rgb]{0.73,0.40,0.53}{{#1}}}
    \newcommand{\ImportTok}[1]{{#1}}
    \newcommand{\DocumentationTok}[1]{\textcolor[rgb]{0.73,0.13,0.13}{\textit{{#1}}}}
    \newcommand{\AnnotationTok}[1]{\textcolor[rgb]{0.38,0.63,0.69}{\textbf{\textit{{#1}}}}}
    \newcommand{\CommentVarTok}[1]{\textcolor[rgb]{0.38,0.63,0.69}{\textbf{\textit{{#1}}}}}
    \newcommand{\VariableTok}[1]{\textcolor[rgb]{0.10,0.09,0.49}{{#1}}}
    \newcommand{\ControlFlowTok}[1]{\textcolor[rgb]{0.00,0.44,0.13}{\textbf{{#1}}}}
    \newcommand{\OperatorTok}[1]{\textcolor[rgb]{0.40,0.40,0.40}{{#1}}}
    \newcommand{\BuiltInTok}[1]{{#1}}
    \newcommand{\ExtensionTok}[1]{{#1}}
    \newcommand{\PreprocessorTok}[1]{\textcolor[rgb]{0.74,0.48,0.00}{{#1}}}
    \newcommand{\AttributeTok}[1]{\textcolor[rgb]{0.49,0.56,0.16}{{#1}}}
    \newcommand{\InformationTok}[1]{\textcolor[rgb]{0.38,0.63,0.69}{\textbf{\textit{{#1}}}}}
    \newcommand{\WarningTok}[1]{\textcolor[rgb]{0.38,0.63,0.69}{\textbf{\textit{{#1}}}}}
    
    
    % Define a nice break command that doesn't care if a line doesn't already
    % exist.
    \def\br{\hspace*{\fill} \\* }
    % Math Jax compatability definitions
    \def\gt{>}
    \def\lt{<}
    % Document parameters
    \title{newnotebook}
    
    
    

    % Pygments definitions
    
\makeatletter
\def\PY@reset{\let\PY@it=\relax \let\PY@bf=\relax%
    \let\PY@ul=\relax \let\PY@tc=\relax%
    \let\PY@bc=\relax \let\PY@ff=\relax}
\def\PY@tok#1{\csname PY@tok@#1\endcsname}
\def\PY@toks#1+{\ifx\relax#1\empty\else%
    \PY@tok{#1}\expandafter\PY@toks\fi}
\def\PY@do#1{\PY@bc{\PY@tc{\PY@ul{%
    \PY@it{\PY@bf{\PY@ff{#1}}}}}}}
\def\PY#1#2{\PY@reset\PY@toks#1+\relax+\PY@do{#2}}

\expandafter\def\csname PY@tok@gd\endcsname{\def\PY@tc##1{\textcolor[rgb]{0.63,0.00,0.00}{##1}}}
\expandafter\def\csname PY@tok@gu\endcsname{\let\PY@bf=\textbf\def\PY@tc##1{\textcolor[rgb]{0.50,0.00,0.50}{##1}}}
\expandafter\def\csname PY@tok@gt\endcsname{\def\PY@tc##1{\textcolor[rgb]{0.00,0.27,0.87}{##1}}}
\expandafter\def\csname PY@tok@gs\endcsname{\let\PY@bf=\textbf}
\expandafter\def\csname PY@tok@gr\endcsname{\def\PY@tc##1{\textcolor[rgb]{1.00,0.00,0.00}{##1}}}
\expandafter\def\csname PY@tok@cm\endcsname{\let\PY@it=\textit\def\PY@tc##1{\textcolor[rgb]{0.25,0.50,0.50}{##1}}}
\expandafter\def\csname PY@tok@vg\endcsname{\def\PY@tc##1{\textcolor[rgb]{0.10,0.09,0.49}{##1}}}
\expandafter\def\csname PY@tok@vi\endcsname{\def\PY@tc##1{\textcolor[rgb]{0.10,0.09,0.49}{##1}}}
\expandafter\def\csname PY@tok@vm\endcsname{\def\PY@tc##1{\textcolor[rgb]{0.10,0.09,0.49}{##1}}}
\expandafter\def\csname PY@tok@mh\endcsname{\def\PY@tc##1{\textcolor[rgb]{0.40,0.40,0.40}{##1}}}
\expandafter\def\csname PY@tok@cs\endcsname{\let\PY@it=\textit\def\PY@tc##1{\textcolor[rgb]{0.25,0.50,0.50}{##1}}}
\expandafter\def\csname PY@tok@ge\endcsname{\let\PY@it=\textit}
\expandafter\def\csname PY@tok@vc\endcsname{\def\PY@tc##1{\textcolor[rgb]{0.10,0.09,0.49}{##1}}}
\expandafter\def\csname PY@tok@il\endcsname{\def\PY@tc##1{\textcolor[rgb]{0.40,0.40,0.40}{##1}}}
\expandafter\def\csname PY@tok@go\endcsname{\def\PY@tc##1{\textcolor[rgb]{0.53,0.53,0.53}{##1}}}
\expandafter\def\csname PY@tok@cp\endcsname{\def\PY@tc##1{\textcolor[rgb]{0.74,0.48,0.00}{##1}}}
\expandafter\def\csname PY@tok@gi\endcsname{\def\PY@tc##1{\textcolor[rgb]{0.00,0.63,0.00}{##1}}}
\expandafter\def\csname PY@tok@gh\endcsname{\let\PY@bf=\textbf\def\PY@tc##1{\textcolor[rgb]{0.00,0.00,0.50}{##1}}}
\expandafter\def\csname PY@tok@ni\endcsname{\let\PY@bf=\textbf\def\PY@tc##1{\textcolor[rgb]{0.60,0.60,0.60}{##1}}}
\expandafter\def\csname PY@tok@nl\endcsname{\def\PY@tc##1{\textcolor[rgb]{0.63,0.63,0.00}{##1}}}
\expandafter\def\csname PY@tok@nn\endcsname{\let\PY@bf=\textbf\def\PY@tc##1{\textcolor[rgb]{0.00,0.00,1.00}{##1}}}
\expandafter\def\csname PY@tok@no\endcsname{\def\PY@tc##1{\textcolor[rgb]{0.53,0.00,0.00}{##1}}}
\expandafter\def\csname PY@tok@na\endcsname{\def\PY@tc##1{\textcolor[rgb]{0.49,0.56,0.16}{##1}}}
\expandafter\def\csname PY@tok@nb\endcsname{\def\PY@tc##1{\textcolor[rgb]{0.00,0.50,0.00}{##1}}}
\expandafter\def\csname PY@tok@nc\endcsname{\let\PY@bf=\textbf\def\PY@tc##1{\textcolor[rgb]{0.00,0.00,1.00}{##1}}}
\expandafter\def\csname PY@tok@nd\endcsname{\def\PY@tc##1{\textcolor[rgb]{0.67,0.13,1.00}{##1}}}
\expandafter\def\csname PY@tok@ne\endcsname{\let\PY@bf=\textbf\def\PY@tc##1{\textcolor[rgb]{0.82,0.25,0.23}{##1}}}
\expandafter\def\csname PY@tok@nf\endcsname{\def\PY@tc##1{\textcolor[rgb]{0.00,0.00,1.00}{##1}}}
\expandafter\def\csname PY@tok@si\endcsname{\let\PY@bf=\textbf\def\PY@tc##1{\textcolor[rgb]{0.73,0.40,0.53}{##1}}}
\expandafter\def\csname PY@tok@s2\endcsname{\def\PY@tc##1{\textcolor[rgb]{0.73,0.13,0.13}{##1}}}
\expandafter\def\csname PY@tok@nt\endcsname{\let\PY@bf=\textbf\def\PY@tc##1{\textcolor[rgb]{0.00,0.50,0.00}{##1}}}
\expandafter\def\csname PY@tok@nv\endcsname{\def\PY@tc##1{\textcolor[rgb]{0.10,0.09,0.49}{##1}}}
\expandafter\def\csname PY@tok@s1\endcsname{\def\PY@tc##1{\textcolor[rgb]{0.73,0.13,0.13}{##1}}}
\expandafter\def\csname PY@tok@dl\endcsname{\def\PY@tc##1{\textcolor[rgb]{0.73,0.13,0.13}{##1}}}
\expandafter\def\csname PY@tok@ch\endcsname{\let\PY@it=\textit\def\PY@tc##1{\textcolor[rgb]{0.25,0.50,0.50}{##1}}}
\expandafter\def\csname PY@tok@m\endcsname{\def\PY@tc##1{\textcolor[rgb]{0.40,0.40,0.40}{##1}}}
\expandafter\def\csname PY@tok@gp\endcsname{\let\PY@bf=\textbf\def\PY@tc##1{\textcolor[rgb]{0.00,0.00,0.50}{##1}}}
\expandafter\def\csname PY@tok@sh\endcsname{\def\PY@tc##1{\textcolor[rgb]{0.73,0.13,0.13}{##1}}}
\expandafter\def\csname PY@tok@ow\endcsname{\let\PY@bf=\textbf\def\PY@tc##1{\textcolor[rgb]{0.67,0.13,1.00}{##1}}}
\expandafter\def\csname PY@tok@sx\endcsname{\def\PY@tc##1{\textcolor[rgb]{0.00,0.50,0.00}{##1}}}
\expandafter\def\csname PY@tok@bp\endcsname{\def\PY@tc##1{\textcolor[rgb]{0.00,0.50,0.00}{##1}}}
\expandafter\def\csname PY@tok@c1\endcsname{\let\PY@it=\textit\def\PY@tc##1{\textcolor[rgb]{0.25,0.50,0.50}{##1}}}
\expandafter\def\csname PY@tok@fm\endcsname{\def\PY@tc##1{\textcolor[rgb]{0.00,0.00,1.00}{##1}}}
\expandafter\def\csname PY@tok@o\endcsname{\def\PY@tc##1{\textcolor[rgb]{0.40,0.40,0.40}{##1}}}
\expandafter\def\csname PY@tok@kc\endcsname{\let\PY@bf=\textbf\def\PY@tc##1{\textcolor[rgb]{0.00,0.50,0.00}{##1}}}
\expandafter\def\csname PY@tok@c\endcsname{\let\PY@it=\textit\def\PY@tc##1{\textcolor[rgb]{0.25,0.50,0.50}{##1}}}
\expandafter\def\csname PY@tok@mf\endcsname{\def\PY@tc##1{\textcolor[rgb]{0.40,0.40,0.40}{##1}}}
\expandafter\def\csname PY@tok@err\endcsname{\def\PY@bc##1{\setlength{\fboxsep}{0pt}\fcolorbox[rgb]{1.00,0.00,0.00}{1,1,1}{\strut ##1}}}
\expandafter\def\csname PY@tok@mb\endcsname{\def\PY@tc##1{\textcolor[rgb]{0.40,0.40,0.40}{##1}}}
\expandafter\def\csname PY@tok@ss\endcsname{\def\PY@tc##1{\textcolor[rgb]{0.10,0.09,0.49}{##1}}}
\expandafter\def\csname PY@tok@sr\endcsname{\def\PY@tc##1{\textcolor[rgb]{0.73,0.40,0.53}{##1}}}
\expandafter\def\csname PY@tok@mo\endcsname{\def\PY@tc##1{\textcolor[rgb]{0.40,0.40,0.40}{##1}}}
\expandafter\def\csname PY@tok@kd\endcsname{\let\PY@bf=\textbf\def\PY@tc##1{\textcolor[rgb]{0.00,0.50,0.00}{##1}}}
\expandafter\def\csname PY@tok@mi\endcsname{\def\PY@tc##1{\textcolor[rgb]{0.40,0.40,0.40}{##1}}}
\expandafter\def\csname PY@tok@kn\endcsname{\let\PY@bf=\textbf\def\PY@tc##1{\textcolor[rgb]{0.00,0.50,0.00}{##1}}}
\expandafter\def\csname PY@tok@cpf\endcsname{\let\PY@it=\textit\def\PY@tc##1{\textcolor[rgb]{0.25,0.50,0.50}{##1}}}
\expandafter\def\csname PY@tok@kr\endcsname{\let\PY@bf=\textbf\def\PY@tc##1{\textcolor[rgb]{0.00,0.50,0.00}{##1}}}
\expandafter\def\csname PY@tok@s\endcsname{\def\PY@tc##1{\textcolor[rgb]{0.73,0.13,0.13}{##1}}}
\expandafter\def\csname PY@tok@kp\endcsname{\def\PY@tc##1{\textcolor[rgb]{0.00,0.50,0.00}{##1}}}
\expandafter\def\csname PY@tok@w\endcsname{\def\PY@tc##1{\textcolor[rgb]{0.73,0.73,0.73}{##1}}}
\expandafter\def\csname PY@tok@kt\endcsname{\def\PY@tc##1{\textcolor[rgb]{0.69,0.00,0.25}{##1}}}
\expandafter\def\csname PY@tok@sc\endcsname{\def\PY@tc##1{\textcolor[rgb]{0.73,0.13,0.13}{##1}}}
\expandafter\def\csname PY@tok@sb\endcsname{\def\PY@tc##1{\textcolor[rgb]{0.73,0.13,0.13}{##1}}}
\expandafter\def\csname PY@tok@sa\endcsname{\def\PY@tc##1{\textcolor[rgb]{0.73,0.13,0.13}{##1}}}
\expandafter\def\csname PY@tok@k\endcsname{\let\PY@bf=\textbf\def\PY@tc##1{\textcolor[rgb]{0.00,0.50,0.00}{##1}}}
\expandafter\def\csname PY@tok@se\endcsname{\let\PY@bf=\textbf\def\PY@tc##1{\textcolor[rgb]{0.73,0.40,0.13}{##1}}}
\expandafter\def\csname PY@tok@sd\endcsname{\let\PY@it=\textit\def\PY@tc##1{\textcolor[rgb]{0.73,0.13,0.13}{##1}}}

\def\PYZbs{\char`\\}
\def\PYZus{\char`\_}
\def\PYZob{\char`\{}
\def\PYZcb{\char`\}}
\def\PYZca{\char`\^}
\def\PYZam{\char`\&}
\def\PYZlt{\char`\<}
\def\PYZgt{\char`\>}
\def\PYZsh{\char`\#}
\def\PYZpc{\char`\%}
\def\PYZdl{\char`\$}
\def\PYZhy{\char`\-}
\def\PYZsq{\char`\'}
\def\PYZdq{\char`\"}
\def\PYZti{\char`\~}
% for compatibility with earlier versions
\def\PYZat{@}
\def\PYZlb{[}
\def\PYZrb{]}
\makeatother


    % Exact colors from NB
    \definecolor{incolor}{rgb}{0.0, 0.0, 0.5}
    \definecolor{outcolor}{rgb}{0.545, 0.0, 0.0}



    
    % Prevent overflowing lines due to hard-to-break entities
    \sloppy 
    % Setup hyperref package
    \hypersetup{
      breaklinks=true,  % so long urls are correctly broken across lines
      colorlinks=true,
      urlcolor=urlcolor,
      linkcolor=linkcolor,
      citecolor=citecolor,
      }
    % Slightly bigger margins than the latex defaults
    
    \geometry{verbose,tmargin=1in,bmargin=1in,lmargin=1in,rmargin=1in}
    
	
	    \title{Adénome de la prostate - Etude des techniques RTUPB et VBPPS}
	    \author{Zineb Bennis \& Fabrice Sougey}
	    \date{ Décembre 2017 }    
		\renewcommand{\contentsname}{Table des matières}
		
    \begin{document}
    
    
    \maketitle
    \thispagestyle{empty}
    
    \newpage
    \tableofcontents
    
    \newpage
    

    
    \section{Considérations}\label{considuxe9rations}

    \subsection{Typage des données}\label{typage-des-donnuxe9es}

\paragraph{}
Pour le typage des données, nous avons fait le choix de garder les
variables numériques par défaut sauf pour certaines variables que nous
avons voulu traiter en tant que Variable Qualitative (= "factor") :
\begin{itemize}
\tightlist
\item Comorbidite\item Porteur sonde\item Caillotage\item Transfusion\item Reprise bloc
\item Anesthésie\item Indication\item Evénement\item Technique\medbreak
\end{itemize}

\paragraph{}
Les variables "QoL" peuvent être considérées comme des variables
ordinales car la "Qualité de vie" est 1 score donné par le patient de 1
à 7. Plus il est élevé, plus le patient est insatisfait. Toutefois nous
avons choisi de le traiter en tant que variable quantitative car nous
allons établir des prédictions pour ce score dans la partie clustering.
Il en va de même pour la variable IPSS. Pour celle-ci la valeur
numérique a encore plus de sens car elle est une addition de différents
scores concernant le patient.

    \subsection{Données manquantes}\label{donnuxe9es-manquantes}

\paragraph{}
    Les individus porteur de sonde ne sont pas concernés par la variable
'Résidu post mictionnel'. En effet, les porteurs de sonde n'urinent pas
par voie naturelle, la sonde faisant le travail mécanique de vider la
vessie. Ils n'ont donc pas de valeurs pour cette feature dans le jeu de
données originel. Ici, plusieurs scénarios sont possibles. On pourrait,
par exemple, enlever la variable 'Résidu post mictionnel' ou alors baser
notre analyse uniquement sur les patients qui ne sont pas porteurs de
sonde.

\paragraph{}
Le jeu de données n'étant pas très important, nous faisons le choix de
ne pas enlever d'individus. Nous enlèverons donc la variable 'Résidu
Post mictionnel' ainsi que la variable 'Qmax' qui n'est pas relevée pour
les individus porteurs de sonde.

\paragraph{}
De plus le volume réséqué n'est disponible que pour la technique
opératoire RTUPB. En effet, cette opération consiste en la résurection
transurétrale de la prostate alors que la technique VBPPS n'enlève pas
de tissu prostatique. Nous pensons que cette donnée va fausser l'analyse
et ne sera pas pertinente dans la discrimination des techniques
opératoires. Nous faisons le choix de la supprimer également.

\paragraph{}
Enfin, nous avons noté que certaines variables sont constantes à tout le
tableau : 'Evenement', 'Transfusion' et 'Reprise Bloc'. Nous faisons le
choix de les supprimer malgré que cela pourrait poser un problème dans
l'éventualité où des nouveaux patients seraient intégrés à l'étude avec
des valeurs différentes de celles aujourd'hui observées.

\paragraph{}
Résumé : Jeu de données complet - moins Résidu post mictionnel - moins
Volume Réséqué - moins Qmax - moins Evenement - moins Transfusion -
moins Reprise Bloc

    \subsubsection{Individus dupliqués ?}\label{individus-dupliquuxe9s}

\paragraph{}
Nous notons que certains individus sont étrangement proches, avec 1
valeur différente sur 1 seule variable ou parfois toutes les variables
avec les mêmes valeurs.

\paragraph{}
    Par exemple, les 3 individus ci-dessous ont les mêmes valeurs pour
toutes les variables.

\paragraph{}

\scalebox{0.8}{
    \begin{tabular}{r|llllllll}
  & Age & Comorbidite & Duree\_Traitement\_Medical & Porteur\_Sonde & IPSS & ... & M18\_QoL & M18\_Qmax\\
\hline
	17 & 84   & 0    & 11   & 1    & 35   & ⋯    & 2    & 17.9\\
	30 & 84   & 0    & 11   & 1    & 35   & ⋯    & 2    & 17.9\\
	36 & 84   & 0    & 11   & 1    & 35   & ⋯    & 2    & 17.9\\
\end{tabular}
}

\paragraph{}
    
    Autre exemple, les individus ci-dessous ont les mêmes valeurs pour
toutes les variables, sauf l'âge et le résidu post-mictionnel.

\paragraph{}
\scalebox{0.8}{
    \begin{tabular}{r|llllllll}
  & Age & Comorbidite & Duree\_Traitement\_Medical & Porteur\_Sonde & IPSS &  ⋯ & M18\_QoL & M18\_Qmax\\
\hline
	2 & 85   & 1    & 82   & 0    & 31   & ⋯ & 1    & 14.4\\
	32 & 86   & 1    & 82   & 0    & 31  & ⋯ & 1    & 14.4\\
\end{tabular}
}

\paragraph{}
    
    Dans une situation réelle nous aurions demandé au client de faire la
lumière sur ce jeu de données et de vérifier qu'il n'y a pas d'erreur.
Ces individus ont également des valeurs éloignées de la moyenne sur
certaines features. Ils vont probablement compromettre le clustering
mais nous faisons le choix de les garder quand même car rien ne nous
indique à ce niveau que ce sont des erreurs.

    \section{Analyse descriptive}\label{analyse-descriptive}

    \subsection{Analyse descriptive -Pré-opératoire}\label{analyse-descriptive---pruxe9-opuxe9ratoire}

    \subsubsection{Distributions}\label{distributions}

    \paragraph{a. Sous-populations}\label{a.-sous-populations}
\paragraph{}
Pour commencer, il nous a paru important de comprendre le découpage de
la population sur certains axes : "Technique" et "Porteur de sonde".

    \begin{center}
    \adjustimage{max size={0.3\linewidth}{0.3\paperheight}}{output_16_0.png}
    
    \adjustimage{max size={0.3\linewidth}{0.3\paperheight}}{output_17_0.png}
    \end{center}

    \paragraph{b. Par variables et par Technique}\label{b.-par-variables-et-par-technique}
\paragraph{}
Pour chaque variable, nous avons fait le choix de visualiser les
distributions globalement et aussi dans les sous-populations des
techniques "RTUPB" et "VBPPS".

    \begin{center}
    \adjustimage{max size={0.3\linewidth}{0.3\paperheight}}{output_19_0.png}
    \end{center}
    
    \begin{center}
    \adjustimage{max size={0.3\linewidth}{0.3\paperheight}}{output_19_1.png}
    \end{center}
    
    \begin{center}
    \adjustimage{max size={0.3\linewidth}{0.3\paperheight}}{output_19_2.png}
    \end{center}
    
    \begin{center}
    \adjustimage{max size={0.3\linewidth}{0.3\paperheight}}{output_19_3.png}
    \end{center}
    
    
    \begin{center}
    \adjustimage{max size={0.3\linewidth}{0.3\paperheight}}{output_19_4.png}
    \end{center}
    
    
    \begin{center}
    \adjustimage{max size={0.3\linewidth}{0.3\paperheight}}{output_19_5.png}
    \end{center}
    
    
    \begin{center}
    \adjustimage{max size={0.3\linewidth}{0.3\paperheight}}{output_19_6.png}
    \end{center}
    
    
    \begin{center}
    \adjustimage{max size={0.3\linewidth}{0.3\paperheight}}{output_19_7.png}
    \end{center}
    
    
    \begin{center}
    \adjustimage{max size={0.3\linewidth}{0.3\paperheight}}{output_19_8.png}
    \end{center}
    
    
    \begin{center}
    \adjustimage{max size={0.3\linewidth}{0.3\paperheight}}{output_19_9.png}
    \end{center}
    
    
    \begin{center}
    \adjustimage{max size={0.3\linewidth}{0.3\paperheight}}{output_19_10.png}
    \end{center}
    
    
    \begin{center}
    \adjustimage{max size={0.3\linewidth}{0.3\paperheight}}{output_19_11.png}
    \end{center}
    
    
    \begin{center}
    \adjustimage{max size={0.3\linewidth}{0.3\paperheight}}{output_19_12.png}
    \end{center}
    
\paragraph{}
    La durée de traitement médical est plus longue pour les patients qui ont
reçu la technique VBPPS. Le temps d'opération est plus long pour les
patients qui ont reçu la technique VBPPS. Il est fort probable que cette
technique demande effectivement plus de temps pour être exécutée. Le
délai d'ablation est moins long pour les patients VBPPS. Aucun cas de
caillotage n'est relevé pour les patients qui ont reçu le traitement
VBPPS.

    \paragraph{c. Porteurs de sonde ou non}\label{c.-porteurs-de-sonde-ou-non}

\paragraph{}
Il nous paraissait intéressant de montrer quelques chiffres en fonction
des porteurs de sonde. En effet cette variable nous semble importante
pour différencier 2 patients.

\begin{itemize}
\tightlist
\item Les porteurs de sonde sont plutôt les patients les plus âgés de notre
population.
\item 
Il n'y a pas 1 choix de technique qui dépende du fait que le patient
porte 1 sonde ou non.
\item 
Par contre, les porteurs de sonde ont l'IPSS maximum et ont aussi tous
donné le même score presque maximal de "qualité de vie" (6 sur 1 échelle
de 1 à 7) montrant la gêne que doit procurer la sonde.
\item 
Tous les patients équipés d'une sonde ont une valeur pour indication
égale à 'Rétention vésicale aigue'. Ces patients sont donc dans
l'incapacité d'uriner ce qui doit être une cause importante de
déclenchement de l'opération.
\end{itemize}


    \begin{center}
    \adjustimage{max size={0.3\linewidth}{0.3\paperheight}}{output_22_0.png}
    \end{center}
    
    
    \begin{center}
    \adjustimage{max size={0.3\linewidth}{0.3\paperheight}}{output_22_1.png}
    \end{center}
    
    
    \begin{center}
    \adjustimage{max size={0.3\linewidth}{0.3\paperheight}}{output_22_2.png}
    \end{center}
    
    


    
    \begin{verbatim}
     Technique
Sonde  1  2
    0 22 20
    1 14 12
    \end{verbatim}

    
    
    \begin{verbatim}
     Indication
Sonde  1  2  3  5  6
    0  0 24  6  8  4
    1 26  0  0  0  0
    \end{verbatim}

    
    \subsubsection{Corrélations}\label{corruxe9lations}

    \paragraph{a. Toute technique}\label{a.-toute-technique}



    \begin{center}
    \adjustimage{max size={0.7\linewidth}{0.7\paperheight}}{output_26_0.png}
    \end{center}
    
\paragraph{}
Nous notons une corrélation positive entre IPSS et QoL. Ainsi qu'entre
la variable "Age" et les variables "QoL" et "IPSS". Plus le patient est
âgé, plus le nombre de symptômes est important et plus son ressenti sur
sa qualité de vie est mauvais.



    \begin{center}
    \adjustimage{max size={0.3\linewidth}{0.3\paperheight}}{output_28_0.png}
    \end{center}
    
    
\paragraph{}
Nous notons aussi une faible corrélation négative entre la "Durée du
traitement médical" et les variables "Délai Ablation" et "Volume
prostatique". Plus le traitement est long moins le délai d'ablation est
important. Et moins la prostate est volumineuse. Cela pourrait vouloir
dire que le traitement pré-opératoire agit sur la taille de la prostate
avec une tendance à la faire diminuer.



    \begin{center}
    \adjustimage{max size={0.3\linewidth}{0.3\paperheight}}{output_30_0.png}
    \end{center}
    
    
    \begin{center}
    \adjustimage{max size={0.3\linewidth}{0.3\paperheight}}{output_30_1.png}
    \end{center}
    
    
    \paragraph{b. Technique RTUPB}\label{b.-technique-rtupb}



    \begin{center}
    \adjustimage{max size={0.7\linewidth}{0.7\paperheight}}{output_32_0.png}
    \end{center}
    
    
\paragraph{}
En réduisant à la sous-population RTUPB, nous constatons les mêmes
corrélations que pour la population globale. Ces corrélations positives
sont légèrement plus fortes. Plus le patient est âgé, plus les symptômes
sont importants et plus le ressenti de qualité de vie est médiocre.
\paragraph{}
Aussi nous retrouvons une corrélation positive entre le Temps
d'opération et le délai d'ablation. Peut-être que l'ablation fait partie
de l'opération et que plus cette étape est longue plus l'opération
complète est longue.



    \begin{center}
    \adjustimage{max size={0.3\linewidth}{0.3\paperheight}}{output_34_0.png}
    \end{center}
    
    
    \paragraph{c. Technique VBPPS}\label{c.-technique-vbpps}



    \begin{center}
    \adjustimage{max size={0.7\linewidth}{0.7\paperheight}}{output_36_0.png}
    \end{center}
    
    
\paragraph{}En réduisant à la sous-population VBPPS, nous étudions les mêmes
corrélations et hypothèses que celles décrites ci-dessus.

    \subsubsection{Analyse en composantes principales}\label{analyse-en-composantes-principales}



    
    \begin{verbatim}
Importance of components:
                          PC1     PC2      PC3     PC4     PC5     PC6     PC7
Standard deviation     28.988 11.1160 10.24931 7.24312 5.31519 1.88519 1.55134
Proportion of Variance  0.727  0.1069  0.09089 0.04539 0.02444 0.00307 0.00208
Cumulative Proportion   0.727  0.8339  0.92481 0.97020 0.99464 0.99772 0.99980
                          PC8
Standard deviation     0.4825
Proportion of Variance 0.0002
Cumulative Proportion  1.0000
    \end{verbatim}
    
\paragraph{}Ainsi les deux premières composantes à elles seules représentent 83\% de
l'information.



    
    
    \begin{center}
    \adjustimage{max size={0.3\linewidth}{0.3\paperheight}}{output_42_1.png}
    \end{center}
    
    \begin{center}
    \adjustimage{max size={0.3\linewidth}{0.3\paperheight}}{output_43_2.png}
    \end{center}
    
    
    \begin{center}
    \adjustimage{max size={0.3\linewidth}{0.3\paperheight}}{output_43_3.png}
    \end{center}
    
    
\paragraph{}
En observant le graphe ci-dessus on peut déduire que la variable qui
contribue le plus dans la PC1 est "Durée du traitement médical". Les
variables qui contribuent le plus à la PC2 sont "Temps Opération" et, à
l'opposé, "IPSS" et "Age".
\paragraph{}
On retrouve les corrélations positives entre les variables "IPSS", "Age"
et "QoL".

    \subsection{Analyse descriptive - Post-opératoire}\label{analyse-descriptive---post-opuxe9ratoire}
\paragraph{}
    Nous avons décidé de ne pas montrer les distributions des variables
post-opératoires prises séparemment car nos essais de boxplots et
barcharts ne nous ont pas permis de dégager de faits intéressants en
post-opératoire.
\paragraph{}
Par contre, afin de voir si nous pouvons déjà dégager plusieurs profils
de guérison parmi tous les patients, nous avons eu l'idée de prendre
plusieurs individus au hasard parmi tout le jeu de données et de faire
un graphique de l'évolution de chacune des variables 'QMax', 'IPSS' et
'Qol' de ces individus en fonction du temps.
\paragraph{}
Nous avons également créé 2 faux individus qui vont représenter les
techniques RTUPB et VBPPS séparemment. Chacun a une valeur pour chaque
variable équivalente à la moyenne de cette variable pour leur
sous-population respective (fonction de la technique). L'idée ici est de
voir si la technique utilisée a 1 impact sur la durée de guérison.
\paragraph{}
Chaque individu est représenté par une courbe sur le même graphique. Un
graphique par indice.

    \subsubsection{Tendances de guérison ?}\label{tendances-de-guuxe9rison}

    \paragraph{a. QMax}\label{a.-qmax}

\paragraph{}
    Le QMax est le débit maximal urinaire. Plus cette valeur est importante,
plus la miction est de bonne qualité. Pour preuve de guérison, on
s'attend donc à voir remonter cette valeur. Au contraire, une valeur qui
stagne ou qui régresse est un signe négatif pour la guérison.



    \begin{center}
    \adjustimage{max size={0.3\linewidth}{0.3\paperheight}}{output_50_0.png}
    \end{center}
    
    
\paragraph{}
On remarque que la technique opératoire utilisée n'a pas d'incidence
forte sur la guérison. Toutefois on peut noter que la technique VBPPS
semble converger plus vite et en moyenne elle procure une légère
meilleure miction à la fin des 18 mois.
\paragraph{}
La courbe moyenne VBPPS commence par 1 régression. Cela est dû à 4
patients qui ont des valeurs de QMax au premier mois égales à presque 3
fois la valeur moyenne de la population globale.
\paragraph{}
Quel que soit le patient, il apparaît que la variable atteint un niveau
qui ne change plus au bout du 9ème mois.



    \begin{center}
    \adjustimage{max size={0.3\linewidth}{0.3\paperheight}}{output_52_0.png}
    \end{center}
    
    
\paragraph{}Ci-dessus la distribution des QMax selon le temps pour les 2 techniques.
On note une stabilisation au bout du 9ème mois.

    \paragraph{b. IPSS}\label{b.-ipss}

\paragraph{}L'IPSS est le "International Prostatic Symptome Score". C'est une
représentation numérique de plusieurs scores en fonction des symptômes
de l'adénome de la prostate.
\paragraph{}
Plus ce chiffre est elevé plus le patient est gené :
\begin{itemize}
\tightlist
\item0 - 7 .......... Peu symptomatique
\item8 - 19 ......... Modérément symptomatique
\item20 - 35 ........ Symptomes sévères
\end{itemize}

\paragraph{}

Pour la guérison, on s'attend à une diminution de ce score.

    \begin{center}
    \adjustimage{max size={0.3\linewidth}{0.3\paperheight}}{output_56_0.png}
    \end{center}
    
    
\paragraph{}
L'IPSS semble diminuer plus rapidement pour la technique RTUPB jusqu'à
atteindre un plancher plus bas que pour la technique VBPPS. L'indice est
rapidement dans la zone "non symptomatique" (en dessous de 8 donc) quel
que soit le patient ou la technique. Il est d'ailleurs très en deça de
l'IPSS pré-opératoire, preuve que l'operation, quelle que soit la
technique, élimine la plupart des symptômes. 

\paragraph{}Ci-dessous les moyennes IPSS :



    \begin{Verbatim}[commandchars=\\\{\}]
Moyenne pré-opératoire - toute population / RTUPB / VBPPS : {\ldots} 27.39706 27.58333 27.1875 
Moyenne post-opératoire 1er mois - toute population /RTUPB / VBPPS : .. 6.955882 6.916667 7 

    \end{Verbatim}



    \begin{center}
    \adjustimage{max size={0.3\linewidth}{0.3\paperheight}}{output_59_0.png}
    \end{center}
    
    
\paragraph{}
Ci-dessus la distribution des IPSS selon le temps pour les 2 techniques.
On note une stabilisation au bout du 9ème mois.

    \paragraph{c. QoL}\label{c.-qol}

\paragraph{}
L'indice 'QoL' représente le score donné par le patient pour sa qualité
de vie. Plus ce score est élevé plus le patient est insatisfait. Ce
score va de 1 à 7.
\paragraph{}
Pour la guérison, on s'attend à voir ce score diminuer.



    \begin{center}
    \adjustimage{max size={0.3\linewidth}{0.3\paperheight}}{output_63_0.png}
    \end{center}
    
    
\paragraph{}Pour le QoL, nous ne notons pas de différence significative entre les 2
techniques. L'évolution est similaire avec une chute importante entre le
1er et le 9ème mois où l'indice atteint un seuil plancher. En moyenne
les patients RTUPB restent légèrement moins satisfaits que les patients
VBPPS.



    \begin{center}
    \adjustimage{max size={0.3\linewidth}{0.3\paperheight}}{output_65_0.png}
    \end{center}
    
    
\paragraph{}Ci-dessus la distribution des QoL selon le temps pour les 2 techniques.
On note une stabilisation au bout du 9ème mois.

    \subsubsection{Corrélations}\label{corruxe9lations}



    \begin{center}
    \adjustimage{max size={0.7\linewidth}{0.7\paperheight}}{output_68_0.png}
    \end{center}
    
    
\paragraph{}Pour rappel, nous n'avions pas la variable "Qmax" en pré-opératoire.
Nous n'avions donc pas pu analyser les corrélations avec les autres
variables. Ici, en post-opératoire, nous pouvons étudier ces
corrélations.
\paragraph{}
En pré-opératoire, nous avions noté des corrélations positives entre
"IPSS" et "QoL" : plus le score IPSS est haut, plus le ressenti du
patient en qualité de vie est médiocre. Sur ce principe et pour une
meilleure visibilité, nous avons choisi de n'étudier les corrélations
que pour les 2 variables "IPSS" et "Qmax".
\paragraph{}
Ces indices étant des scores liés par une notion de temps et censés
évoluer dans le même sens en fonction du temps, les corrélations
semblent évidentes entre les mêmes indices à différents moments. Plus le
temps passe et plus les indices sont corrélés entre eux. C'est un signe
de convergence et d'orientation dans le sens de la guérison.
\paragraph{}
Les variables Qmax sont corrélées négativement avec les variables "IPSS"
et "QoL" : plus les symptômes vont diminuer et la vie du patient
s'améliorer, plus la miction deviendra bonne. Une miction qui s'améliore
est un signe évident de guérison du patient.

    \subsubsection{Conclusion analyse exploratoire post-opération}\label{conclusion-analyse-exploratoire-post-opuxe9ration}

    De cette analyse exploratoire post-opération, on peut en conclure
plusieurs choses : 
\begin{itemize}
\tightlist
\item l'opération est bénéfique pour le patient qui peut
espérer fortement une guérison
\item une grande partie des patients aura
guerri après le 9ème mois
\item il existe plusieurs profils qui auront une
guérison plus ou moins rapide
\item la technique opératoire ne semble pas
avoir d'incidence significative sur la durée de guérison
\end{itemize}

    \section{Clustering des données
RTUPB-VBPPS}\label{clustering-des-donnuxe9es-rtupb-vbpps}

    \subsection{Extraction des profils types de patients à partir des
données
pré-opératoires}\label{extraction-des-profils-types-de-patients-uxe0-partir-des-donnuxe9es-pruxe9-opuxe9ratoires}

    \subsubsection{CAH}\label{cah}

\paragraph{}
La Classification Ascendante Hiérarchique est une bonne première
approche pour aider à la détermination du nombre de classes idéal.
Notamment par le biais du dendrogramme qui permet de visualiser des
intuitions.
\paragraph{}
Nous prenons bien soin de retirer la variable "Technique" de notre
recherche de clustering car nous allons chercher à étudier cette
variable plus tard dans l'analyse. Il ne serait pas correct de
l'utiliser pour calculer les dissimilarités.
\paragraph{}
Le calcul de la matrice des dissimilarités se fera à l'aide de la
fonction R 'Daisy' et de la distance de Gower. Cette distance permet de
calculer des similarités et dissimilarités sur des matrices hétérogènes
comme c'est le cas dans notre jeu de données. En effet, nous avons des
variables quantitatives et des variables qualitatives.



    
    \begin{verbatim}
2278 dissimilarities, summarized :
   Min. 1st Qu.  Median    Mean 3rd Qu.    Max. 
 0.0000  0.2784  0.3832  0.3716  0.4639  0.7351 
Metric :  mixed ;  Types = I, N, I, N, I, I, I, I, N, N, I, I, N 
Number of objects : 68
    \end{verbatim}

    


    \begin{center}
    \adjustimage{max size={0.7\linewidth}{0.7\paperheight}}{output_77_0.png}
    \end{center}
    
    
\paragraph{}
En regardant le dendrogramme, nous pouvons évaluer un nombre de classes
de 3 ou 4.

    \subsubsection{PAM}\label{pam}

\paragraph{}
Effectuons maintenant le test de silhouette avec l'algorithme PAM
(Partitioning Around Medoids) pour confirmer ou infirmer notre intuition
déduite de la CAH.



    \begin{center}
    \adjustimage{max size={0.3\linewidth}{0.3\paperheight}}{output_81_0.png}
    \end{center}
    
\paragraph{}    
L'analyse des silhouettes n'est pas concluante et ne vient pas confirmer
notre intuition apportée avec l'aide du dendrogramme. En effet, il
faudrait un nombre très important de classes pour avoir des valeurs
silhouette proches de 1. Pour être au delà de 0.7, valeur communément
admise comme silhouette moyenne acceptable, il faut avoir plus de 10
classes. Par la suite, il sera très difficile d'interpréter un si grand
nombre de classes.
\paragraph{}
La valeur silhouette n'est qu'une clé pour aider au choix du nombre de
classes. En général, il n'est pas intéressant, voire contre-productif,
d'avoir 1 nombre de clusters supérieur à 10\% de la population. Dans
notre cas, cela signifie que nous ne devrions pas aller au-delà de 6
classes.
\paragraph{}
Jusqu'à 6 classes, les valeurs silhouettes les plus importantes sont
pour 2 et 3 classes. En recoupant avec l'intuition apportée par le
dendrogramme, nous pensons que 3 classes est un bon choix.
\paragraph{}
De plus, pour 3 classes, nous avons une répartition plutôt homogène des
individus.
\paragraph{}


    \begin{tabular}{lllll}
 size & max\_diss & av\_diss & diameter & separation\\
\hline
	 24        & 0.3380102 & 0.1985036 & 0.4203080 & 0.2659938\\
	 26        & 0.2702409 & 0.1395546 & 0.3941044 & 0.2741430\\
	 18        & 0.3195286 & 0.2347373 & 0.5184103 & 0.2659938\\
\end{tabular}

\paragraph{}
    
    Nous faisons donc le choix de 3 profils de patients.
\paragraph{}
    Ci-dessous, une visualisation en 2D des 3 clusters selon les 2 premières
composantes principales.



    \begin{center}
    \adjustimage{max size={0.5\linewidth}{0.5\paperheight}}{output_87_0.png}
    \end{center}
    
    
    \subsubsection{Caractérisation des patients
médoids}\label{caractuxe9risation-des-patients-muxe9doids}

\paragraph{}
Ci-dessous les 3 patients représentant les 3 classes.

\paragraph{}

\scalebox{0.5}{
    \begin{tabular}{r|lllllllllllll}
  & Age & Comorbidite & Duree\_Traitement\_Medical & Porteur\_Sonde & IPSS & QoL & PSA & Volume\_Prostatique & Indication & Anesthesie & Temps\_Operation & Delai\_Ablation & Caillotage\\
\hline
	14 & 64   & 0    & 13   & 0    & 29   & 5    & 0.18 & 56   & 2    & 1    & 35   & 2    & 0   \\
	55 & 75   & 0    & 38   & 1    & 35   & 6    & 3.70 & 52   & 1    & 1    & 39   & 2    & 0   \\
	10 & 68   & 0    & 13   & 0    & 20   & 4    & 7.84 & 45   & 2    & 2    & 34   & 2    & 0   \\
\end{tabular}
}

    
    \paragraph{a. Représentant profil 1 (individu no
14)}\label{a.-repruxe9sentant-profil-1-individu-no-14}

\begin{itemize}
\tightlist
\item
Le premier représentant est un patient jeune (64 ans et donc sur le
premier quartile de la distribution des âges). Il a eu une durée de
traitement médical plutôt courte.
\item
Il n'était pas porteur de sonde et pourtant il a un score IPSS de 29 qui
signale des sympômes sévères et un score de qualité de vie ressentie
haut (5 sachant que 6 est le maximum observé).
\item
Il a très peu d'antigène prostatique dans le sang, montrant un signe
positif dans la maladie bénigne de l'adénome de la prostate.
\item
Par contre, il a 1 volume prostatique plus haut que la moyenne (56 alors
que la moyenne est de 53 environ). Il est opéré sous anesthésie
loco-régionale, suite à l'échec du traitement médical (indication = 2).
\end{itemize}

    \paragraph{b. Représentant profil 2 (individu no
55)}\label{b.-repruxe9sentant-profil-2-individu-no-55}

\begin{itemize}
\tightlist
\item
Le deuxième représentant est un patient plus vieux que le premier
représentant, 75 ans, sans faire partie des patients les plus âgés. Il
est légèrement au dessus de la moyenne qui se situe aux alentours des 71
ans.
\item
Il a eu une durée de traitement au dessus de la moyenne (38 mois).
\item
Il est porteur de sonde et a donc 1 IPSS et 1 Qol les plus hauts
observés.
\item
Il a 1 taux d'antigène prostatique dans le sang supérieur à la moyenne
(3.7 pour 2,7) et un volume prostatique dans la moyenne.
\item
Il a été opéré sous anesthésie loco-régionale, suite à une rétention
vésicale. Sa rétention est d'ailleurs la cause pour laquelle il porte
une sonde.
\end{itemize}

    \paragraph{c. Représentant profil 3 (individu no
10)}\label{c.-repruxe9sentant-profil-3-individu-no-10}

\begin{itemize}
\tightlist
\item
Le troisième représentant est un patient qui a 1 âge correspondant à la
médiane des âges de la population globale. Il a eu une durée de
traitement médical plutôt courte.
\item
Il ne porte pas de sonde et il a 1 score de qualité de vie relativement
bas (4 est le plus petit indice donné en 'QoL' en pré-opératoire), donc
un ressenti de qualité de vie moins pire que les autres patients.
\item
Il a 1 taux d'antigène prostatique dans le sang très élevé (la valeur la
plus élevée des observations) et un volume prostatique sous la moyenne.
Il a été opéré sous anesthésie générale, suite à l'échec de son
traitement médical (indication = 2).
\end{itemize}

    \paragraph{d. Tout représentant}\label{d.-tout-repruxe9sentant}

\begin{itemize}
\tightlist
\item
  Représentant 1 : jeune / non porteur de sonde / sympômes importants /
  volume prostatique élevé
\item
  Représentant 2 : âge moyen / porteur de sonde / sympômes importants
\item
  Représentant 3 : âge moyen / non porteur de sonde / PSA élevé
\end{itemize}

Aucun des représentants ne présente de comorbidité. Le temps d'opération
pour chacun est assez proche de la moyenne (38) et loin des minima (13)
et maxima (60).

    \subsubsection{Visualisation des distributions du Q\_max à 12 mois
pour les différentes
classes}\label{visualisation-des-distributions-du-q_max-uxe0-12-mois-pour-les-diffuxe9rentes-classes}

\paragraph{}
Notre clustering a été calculé sans la variable 'Technique'. Cela
signifie que les profils de patients ne tiennent pas compte de cette
information. Si on ré-intègre cette variable et qu'on affiche leur
distribution pour le QMax à 12 mois de chaque cluster, on pourra voir
quelle est la technique la plus synonyme de succès selon ce cluster.



    \begin{center}
    \adjustimage{max size={0.3\linewidth}{0.3\paperheight}}{output_97_0.png}
    \end{center}
    
    
    \begin{center}
    \adjustimage{max size={0.3\linewidth}{0.3\paperheight}}{output_97_1.png}
    \end{center}
    
    
    \begin{center}
    \adjustimage{max size={0.3\linewidth}{0.3\paperheight}}{output_97_2.png}
    \end{center}
    
    
\paragraph{}Un individu qui appartiendrait au cluster 3 aurait une miction de
meilleure qualité à 12 mois si il a subi la technique RTUPB.
Inversement, un individu qui appartiendrait au cluster 2 ou au cluster 1
aurait une meilleure miction à 12 mois si il subi la technique
opératoire VBPPS.
\paragraph{}
On peut déduire qu'un patient qui aurait les caractéristiques du profil
\#3 (âge moyen / non porteur de sonde / PSA élevé) devrait être orienté
vers la technique RTUPB.
\paragraph{}
Un patient qui aurait les caractéristiques du profil \#1 (jeune / non
porteur de sonde / sympômes importants / volume prostatique élevé) ou
\#2 (âge moyen / porteur de sonde / sympômes importants) devrait être
orienté vers la technique opératoire RTUPB.

    \subsection{Extraction des profils types des patients à partir des
données
post-opératoires¶}\label{extraction-des-profils-types-des-patients-uxe0-partir-des-donnuxe9es-post-opuxe9ratoires}

\subsubsection{CAH}\label{cah}

\paragraph{}A l'aide des séries temporelles crées pour la partie exploratoire, nous
créons une matrice qui assemble ces séries temporelles.
\paragraph{}
On calcule ensuite une matrice de dissimilarités en utilisant la
distance euclidienne.



    
    \begin{verbatim}
   Min. 1st Qu.  Median    Mean 3rd Qu.    Max. 
   0.00   12.77   23.13   32.55   46.56  126.02 
    \end{verbatim}

    


    \begin{center}
    \adjustimage{max size={0.7\linewidth}{0.7\paperheight}}{output_102_0.png}
    \end{center}
    
    
\paragraph{}En regardant le dendrogramme, nous pouvons évaluer un nombre de classes
de 2 ou 3.

    \subsubsection{PAM}\label{pam}



    \begin{center}
    \adjustimage{max size={0.3\linewidth}{0.3\paperheight}}{output_105_0.png}
    \end{center}
    
    
\paragraph{}
Les valeurs silhouette justifieraient plutôt de faire le choix de 2
profils. Avec 3 profils, la valeur silhouette reste assez haute. Au
delà, les valeurs sont mauvaises. L'intuition que nous avons eue avec le
dendrogramme est vérifiée.
\paragraph{}
    Ci-dessous le graphique avec 2 classes.
\paragraph{}


    \begin{tabular}{lllll}
 size & max\_diss & av\_diss & diameter & separation\\
\hline
	 60       & 44.29086 & 15.21593 & 68.52707 & 21.16955\\
	  8       & 34.02999 & 18.07269 & 46.91716 & 21.16955\\
\end{tabular}


    
    \begin{center}
    \adjustimage{max size={0.3\linewidth}{0.3\paperheight}}{output_108_1.png}
    \end{center}
    
    
    En prenant 2 clusters, il nous semble que l'on intègre au premier
cluster quelques valeurs qui pourraient former leur propre groupe. Si on
essaye avec 3 groupes, cette intuition est confirmée ci-dessous. Les
individus isolés forment bien le troisième cluster.

\bigbreak

    \begin{tabular}{lllll}
 size & max\_diss & av\_diss & diameter & separation\\
\hline
	 56        & 44.290857 & 13.218910 & 68.527075 & 21.16955 \\
	  8        & 34.029987 & 18.072695 & 46.917161 & 21.16955 \\
	  4        &  2.964793 &  1.783514 &  3.525621 & 37.46345 \\
\end{tabular}

\bigbreak
    
    \begin{center}
    \adjustimage{max size={0.3\linewidth}{0.3\paperheight}}{output_110_1.png}
    \end{center}
    
    
    La distribution des groupes n'est pas homogène, avec 2 ou 3 groupes. En
effet le premier groupe est très peuplé.
\medbreak
Nous faisons finalement le choix de partir avec 3 profils de guérison.

    \subsubsection{Caractérisation des profils de guérison
médoids}\label{caractuxe9risation-des-profils-de-guuxe9rison-muxe9doids}


\paragraph{}
    \begin{tabular}{r|lllllll}
  & M1\_IPSS & M3\_IPSS & M6\_IPSS & M9\_IPSS & M12\_IPSS & M15\_IPSS & M18\_IPSS\\
\hline
	9 & 6 & 2 & 2 & 2 & 2 & 2 & 2\\
	49 & 6 & 2 & 1 & 1 & 1 & 1 & 1\\
	42 & 9 & 6 & 3 & 3 & 3 & 3 & 3\\
\end{tabular}

\paragraph{}
    
    \begin{tabular}{r|lllllll}
  & M1\_QoL & M3\_QoL & M6\_QoL & M9\_QoL & M12\_QoL & M15\_QoL & M18\_QoL\\
\hline
	9 & 3 & 2 & 2 & 2 & 2 & 2 & 2\\
	49 & 3 & 2 & 1 & 1 & 1 & 1 & 1\\
	42 & 3 & 3 & 1 & 1 & 1 & 1 & 1\\
\end{tabular}

\paragraph{}
    
    \begin{tabular}{r|lllllll}
  & M1\_Qmax & M3\_Qmax & M6\_Qmax & M9\_Qmax & M12\_Qmax & M15\_Qmax & M18\_Qmax\\
\hline
	9 & 12.4 & 15.7 & 14.9 & 15.1 & 15.1 & 16.0 & 16.0\\
	49 & 38.4 & 28.9 & 39.6 & 40.1 & 39.4 & 39.2 & 39.0\\
	42 & 50.0 & 35.0 & 12.0 & 12.3 & 12.4 & 12.8 & 12.9\\
\end{tabular}


    
    \paragraph{a. Représentant profil 1 (individu no
9)}\label{a.-repruxe9sentant-profil-1-individu-no-9}

\begin{itemize}
\tightlist
\item
Cet individu a 1 IPSS à 6 au premier mois, IPSS 'peu symptomatique'. Il
se stabilise rapidement à 2 au bout du troisième mois.
\item
Son score de qualité de vie commence à 3 (équivalent aux autres profils)
passe à 2 au bout du troisième mois pour ne plus bouger. Il reste plus
haut que les scores des autres profils.
\item
Son Qmax est à 12.4 au premier mois et atteint 16 au 18eme mois après
une chute au 6eme mois.
\end{itemize}

    \paragraph{b. Représentant profil 2 (individu no
49)}\label{b.-repruxe9sentant-profil-2-individu-no-49}

\begin{itemize}
\tightlist
\item
Le second profil a un IPSS 'Peu symptomatique' de 6 au premier mois qui
descend à 2 au troisième mois. Et se stabilise à 1 à partir du 6eme
mois.
\item
Son score de qualité de vie commence à 3, descend à 2 au troisième mois.
Et se stabilise à 1 au 6ème mois.
\item
Le Qmax est à 38.4 au premier mois. Il évolue peu au cours du temps. Il
termine à 39 au 18ème mois.
\end{itemize}

    \paragraph{c. Représentant profil 3 (individu no
42)}\label{c.-repruxe9sentant-profil-3-individu-no-42}


\begin{itemize}
\tightlist
\item
Il a 1 score IPSS au premier mois plus haut que les autres profils.
Cette valeur place le profil dans la classe 'Modérément symptomatique'
montrant qu'il reste encore des symptomes au premier mois après
l'opération. L'indice atteint sa valeur plancher et stabilisée à 3 au
bout du sixième mois. Son score IPSS reste le plus haut des 3 profils.
\item
L'indice QoL atteint sa valeur plancher 1 au sixième mois.
\item
Enfin, son Qmax commence plus haut que les autres profils, chute au 3eme
et 6mois pour atteindre la valeur de 12. Valeur qui remonte doucement
jusqu'à 12.9 pour le 18eme mois. La valeur finale de l'indice reste plus
basse que les autres profils. Le patient aura une moins bonne miction
après le 18ème mois.
\end{itemize}

    \subsection{Classification
supervisée}\label{classification-supervisuxe9e}

    \subsubsection{Arbres de
régression}\label{arbres-de-ruxe9gression}

\paragraph{}Tous nos arbres sont des arbres de régression car ils prédisent des
variables quantitatives, y compris la variable "QoL" que nous avons
choisi de traiter comme numérique malgré qu'elle soit réellement de type
"ordinal".

    \paragraph{a. IPSS}\label{a.-ipss}



    \begin{center}
    \adjustimage{max size={0.3\linewidth}{0.3\paperheight}}{output_121_0.png}
    \end{center}
    
\paragraph{}
On retrouve les variables "Durée traitement médical" et "Temps
opération" comme critères important de ségrégation. Ce sont 2 variables
que nous avions noté d'importance dans la partie analyse exploratoire.
\paragraph{}
    Pour évaluer la performance de l'arbre, nous générons une prédiction
pour 10 individus pris au hasard et nous les comparons avec les valeurs
réelles observées. Ci-dessous les résultats :

\paragraph{}

    \begin{tabular}{r|ll}
  & Prédiction & Observé\\
\hline
	65 & 13.000000 & 13       \\
	33 &  1.000000 &  1       \\
	59 &  0.000000 &  0       \\
	60 &  3.333333 &  4       \\
	39 & 13.000000 & 13       \\
	26 &  2.000000 &  2       \\
	10 &  2.000000 &  2       \\
	58 &  3.333333 &  3       \\
	19 &  1.000000 &  1       \\
	4 &  2.000000 &  2       \\
\end{tabular}
    
    \paragraph{b. QMax}\label{b.-qmax}



    \begin{center}
    \adjustimage{max size={0.3\linewidth}{0.3\paperheight}}{output_126_0.png}
    \end{center}
    
    
\paragraph{}Pour évaluer la performance de l'arbre, nous générons une prédiction
pour 10 individus pris au hasard et nous les comparons avec les valeurs
réelles observées. Ci-dessous les résultats :

\paragraph{}

    \begin{tabular}{r|ll}
  & Prédiction & Observé\\
\hline
	22 & 15.10000 & 15.1    \\
	48 & 12.45000 & 12.6    \\
	18 & 16.20000 & 15.9    \\
	39 &  3.70000 &  3.4    \\
	31 & 16.20000 & 16.1    \\
	17 & 18.50000 & 18.1    \\
	36 & 18.50000 & 18.1    \\
	56 & 12.92500 & 13.2    \\
	55 & 18.56667 & 18.8    \\
	63 & 44.76667 & 40.7    \\
\end{tabular}


    
    \paragraph{c. QoL}\label{c.-qol}



    \begin{center}
    \adjustimage{max size={0.3\linewidth}{0.3\paperheight}}{output_130_0.png}
    \end{center}
    
    
\paragraph{}
Pour évaluer la performance de l'arbre, nous générons une prédiction
pour 10 individus pris au hasard et nous les comparons avec les valeurs
réelles observées. Ci-dessous les résultats :

\paragraph{}

    \begin{tabular}{r|ll}
  & Prédiction & Observé\\
\hline
	60 & 1.6666667 & 2        \\
	63 & 0.7500000 & 0        \\
	59 & 0.3333333 & 1        \\
	44 & 1.0000000 & 1        \\
	61 & 0.3333333 & 0        \\
	33 & 1.0000000 & 1        \\
	36 & 2.0000000 & 2        \\
	21 & 1.0000000 & 1        \\
	64 & 0.7500000 & 1        \\
	2 & 1.0000000 & 1        \\
\end{tabular}


    
    \subsubsection{Prédiction de la classe de
guérison}\label{pruxe9diction-de-la-classe-de-guuxe9rison}

\paragraph{}
L'idée ici est de reprendre la classification de guérison trouvée en
partie 3.2 et de créer un arbre de décision basé sur les variables
pré-opératoire permettant de prédire cette classification. On parlera
d'arbre de classification car la variable à prédire est une classe
(variable qualitative donc).



    \begin{center}
    \adjustimage{max size={0.3\linewidth}{0.3\paperheight}}{output_135_0.png}
    \end{center}
    
\paragraph{}    
Pour évaluer la performance de l'arbre, nous générons une prédiction
pour 10 individus pris au hasard et nous les comparons avec les valeurs
réelles observées. Ci-dessous les résultats :

\paragraph{}

    \begin{tabular}{r|llll}
  & Observé & Prédiction classe = 1 & Prédiction classe = 2 & Prédiction classe = 3\\
\hline
	44 & 1 & X & 0 & 0\\
	68 & 3 & 0 & 0 & X\\
	65 & 1 & X & 0 & 0\\
	27 & 1 & X & 0 & 0\\
	8 & 1 & X & 0 & 0\\
	34 & 1 & X & 0 & 0\\
	14 & 1 & X & 0 & 0\\
	30 & 1 & X & 0 & 0\\
	23 & 1 & X & 0 & 0\\
	59 & 1 & X & 0 & 0\\
\end{tabular}


    

    % Add a bibliography block to the postdoc
    
    
    
    \end{document}
